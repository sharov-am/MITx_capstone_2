\section{Linear Classifier}
Feature vectors $x$, labels $y$
\begin{align*}
x \in \mathbb{R}^d\\
y \in \{-1,1\}
\end{align*}
Training set
\begin{align*}
S_n = \{(x^{(i)}, y^{(i)}), i=1,...,n\}
\end{align*}

Classifier
\begin{align*}
h: \mathbb{R}^d \rightarrow \{-1,1\}\\
\chi^{+} = \{x \in \mathbb{R}^d: h(x) =1\}\\
\chi^{-} = \{x \in \mathbb{R}^d: h(x) =-1\}
\end{align*}

Training error
\begin{align*}
\varepsilon_n(h)= \frac{1}{n} \sum_{i=1}^n \textbf{1}\{h(x^{(i)}) \neq y^{(i)} \}
\end{align*}

Test error (over disjoint set of examples)

\begin{align*}
\varepsilon(h)
\end{align*}

Set of classifiers
\begin{align*}
h \in H
\end{align*}

\subsection{Linear classifiers through origin}

Set of all points that satisfies a line through the origin.

\begin{align*}
\theta &= \begin{bmatrix}
           \theta_{1} \\
           \theta_{2}
         \end{bmatrix}\\
X &= \begin{bmatrix}
           x_{1} \\
           x_{2}
         \end{bmatrix}\\
\end{align*}

Decision Boundary
\begin{align*}
\{x&: \theta_1 x_1 + \theta_2 x_2 = 0\}\\
\{x&: \theta \cdot X=0\}
\end{align*}

Linear Classifier through origin
\begin{align*}
h(x,\theta)=sign(\theta \cdot X)\\
\Theta \in \mathbb{R}^d
\end{align*}

\subsection{Linear classifiers}
General linear Classifier (with Intercept)
\begin{align*}
\theta &= \begin{bmatrix}
           \theta_{1} \\
           \theta_{2}
         \end{bmatrix}\\
X &= \begin{bmatrix}
           x_{1} \\
           x_{2}
         \end{bmatrix}\\
\end{align*}

Decision Boundary
\begin{align*}
\{x&: \theta \cdot X + \theta_0 = 0\}
\end{align*}

Linear Classifier through origin
\begin{align*}
h(x,\Theta, \theta_0)=sign(\theta \cdot X + \theta_0)\\
\theta \in \mathbb{R}^d\\
\theta_0 \in \mathbb{R}
\end{align*}

\subsection{Linear Separation}

Traning examples $S_n = \{(x^{(i)}, y^{(i)}), i=1,...,n\}$ are linear separable if there exists a parameter vector $\hat{\theta}$ and offset parameter $\hat{\theta}_0$ such that $y^{(i)}(\hat{\theta} \cdot x^{(i)} + \hat{\theta}_0 )>0$ for all $i=1,\cdots,n$.

\begin{align*}
(\hat{\theta} \cdot x^{(i)})>0
	\begin{cases}
		 y^{(i)}>0 \text{ and } \theta \cdot x^{(i)} >0\\
		 y^{(i)}<0 \text{ and } \theta \cdot x^{(i)} <0\\
	\end{cases}
\end{align*}

$y^{(i)}(\theta \cdot x^{(i)})>0$ if label and classified result match. This leads to a new definition of the \textbf{Training error}:

\begin{align*}
\varepsilon_n(\theta)&= \frac{1}{n} \sum_{i=1}^n \textbf{1}\{y^{(i)}(\theta \cdot x^{(i)}) \leq 0\}\\
\varepsilon_n(\theta,\theta_0)&= \frac{1}{n} \sum_{i=1}^n \textbf{1}\{y^{(i)}(\theta \cdot x^{(i)} + \theta_0) \leq 0\}
\end{align*}

\subsection{Perceptron through Origin}
\textbf{Perceptron}$\displaystyle \left(\big \{ (x^{(i)}, y^{(i)}), i=1,...,n\big \} , T \right):$
\begin{enumerate}[\indent {}]
	\item initialize  $\theta=0$ (vector);
	\begin{enumerate}[\indent {}]
		\item for $t=1,\cdots,T$ do
		\begin{enumerate}[\indent {}]
			\item for $i=1,\cdots,n$ do
			\begin{enumerate}[\indent {}]
				\item if $y^{(i)}(\theta \cdot x^{(i)})\leq 0$ then 
				\item update $\theta = \theta + y^{(i)}x^{(i)}$
			\end{enumerate}
		\end{enumerate}
	\end{enumerate}
\end{enumerate}

\subsection{Perceptron with Offset}

\textbf{Perceptron}$\displaystyle \left(\big \{ (x^{(i)}, y^{(i)}), i=1,...,n\big \} , T \right):$
\begin{enumerate}[\indent {}]
	\item initialize  $\theta=0$ (vector); $\theta_0=0$ (scalar)
	\begin{enumerate}[\indent {}]
		\item for $t=1,\cdots,T$ do
		\begin{enumerate}[\indent {}]
			\item for $i=1,\cdots,n$ do
			\begin{enumerate}[\indent {}]
				\item if $y^{(i)}(\theta \cdot x^{(i)} + \theta_0)\leq 0$ then
				\item update $\theta = \theta + y^{(i)}x^{(i)}$
				\item update $\theta_0=\theta_0 + y^{(i)}$
			\end{enumerate}
		\end{enumerate}
	\end{enumerate}
\end{enumerate}

\subsection{Margin Boundary}

The Margin Boundary is the set of points  $x$  which satisfy $\theta \cdot x + \theta _0= \pm 1$. So, the distance (with signed direction) from the decision boundary to the margin boundary is $\displaystyle \frac{1}{\mid \mid \theta \mid \mid }$.

$\frac{y^{(i)}(\theta \cdot x^{(i)} + \theta _0)}{\mid \mid \theta \mid \mid }=\frac{1}{\mid \mid \theta \mid \mid }.$

\textbf{Hinge Loss (agreement)}
\begin{align*}
Agreement &= z = y^{(i)}(\theta \cdot x^{(i)} + \theta _0)\\
Loss_h(z) &= 
\begin{cases}
		 0 \text{ if } z \geq 1\\
		 1-z \text{ if } z < 1
\end{cases} 
\end{align*}

\textbf{Regularization} means pushing out the margin boundaries by adding $max(\frac{1}{\mid \mid \theta \mid \mid })$ or $min(\frac{1}{2}\mid \mid \theta \mid \mid^2)$ to the objective function.

Alternatively, the sum of the hinge losses can be calculated by
$\sum _{i=1}^{n} \max \{ 0, 1 - y^{(i)} (\theta \cdot x^{(i)} + \theta _0)\}$

\textbf{Objective Function}

Objective function = average loss + regularization\\

Objective function is minimized, learning becomes an optimization problem. Using hinge loss and margin boundaries is called \textbf{Support Vector Machine} or \textbf{Large margin linear classification}:

\begin{align*}
J(\theta , \theta _0) = \frac{1}{n} \sum _{i=1}^{n} \text {Loss}_ h (z) + \frac{\lambda }{2} \mid \mid \theta \mid \mid ^2.
\end{align*}

Where $\lambda > 0$ is called the regularization parameter that regulates how important the margin boundaries are in comparison to the average hinge loss.

\textbf{Cost:} is an averaged loss.

\subsection{Gradient Descent}

Assume $\theta \in \mathbb{R}$ the goal is to find $\theta$ that minimizes $J(\theta , \theta _0) = \frac{1}{n} \sum _{i=1}^{n} \text {Loss}_ h (y^{(i)} (\theta \cdot x^{(i)} + \theta _0 )) + \frac{\lambda }{2} \mid \mid \theta \mid \mid ^2$ through gradient descent.\\

In other words, we will
\begin{itemize}
\item Start $\theta$ at an arbitrary location: $\theta \leftarrow \theta _{start}$
\item Update $\theta$ repeatedly with $\theta \leftarrow \theta - \eta \frac{\partial J(\theta , \theta _0)}{\partial \theta }$ until $\theta$ does not change significantly.
\end{itemize}

Where $\eta >0$ is called the stepsize or \textbf{learning parameter}.

\subsection{Stochastic Gradient Descent}

\begin{align*}
J(\theta , \theta _0) &= \frac{1}{n} \sum _{i=1}^{n} \text {Loss}_ h (z) + \frac{\lambda }{2} \mid \mid \theta \mid \mid ^2\\
&= \frac{1}{n} \sum _{i=1}^{n}\big [ \text {Loss}_ h (z) + \frac{\lambda }{2} \mid \mid \theta \mid \mid ^2 \big ]
\end{align*}

With stochastic gradient descent, we choose $i \in \big \{ 1,...,n \big \}$ at random and update $\theta$ such that\\

\begin{align*}
\theta \leftarrow \theta - \eta \nabla _{\theta } \big [\text {Loss}_ h(z) + \frac{\lambda }{2}\mid \mid \theta \mid \mid ^2 \big ]
\end{align*}
