\section{Regression and classification}

Classification:
\begin{align*}
S_n = \{(x^{(t)},x^{(t)}))|t=1,\cdots,n\}\\
x^{(t)} \in \mathbb{R}^d, y^{(t)} \in \{-1,1\}
\end{align*}

Regression:
\begin{align*}
y^{(t)} \in \mathbb{R}\\
f(x,\theta,\theta_0) &= \sum_{i=1}^d (\theta_i x_i + \theta_0) = \\
&=\theta \cdot x + \theta_0
\end{align*}

\subsection{Objective for linear regression}

The empirical risk  $R_n$  is defined as
\begin{align*}
R_ n(\theta ) = \frac{1}{n} \sum _{t=1}^{n} \text {Loss}(y^{(t)} - \theta \cdot x^{(t)})
\end{align*}

where  $(x^{(t)},y^{(t)})$  is the  $t$th training example (and there are $n$ in total), and  Loss  is some loss function, such as hinge loss.

Possible to get \textbf{closed form solution} for gradient because function is concave. Only possible if the $dxd$ matrix $A$ is invertible. Computationally expensive if dimensions are very high like in bag of words approach.

\begin{align*}
\nabla R_ n(\theta ) &= A\theta - b (=0)\\
&= A^{-1}b\\
\text {where } \\
A &= \frac{1}{n} \sum _{t=1}^{n} x^{(t)} ( x^{(t)})^ T\\
b &= \frac{1}{n} \sum _{t=1}^{n} y^{(t)} x^{(t)}
\end{align*}

$b$ is a vector with dimensionality $d$.
\subsection{Gradient based Approach}
Nudge gradient in the opposite direction to find (local) minima.

\begin{align*}
\nabla_{\theta}(y^{(t)} - \theta x^{(t)})^2 / 2 =\\
= (y^{(t)} - \theta x^{(t)}) \nabla_{\theta}(y^{(t)} - \theta x^{(t)}) = \\
= -(y^{(t)} - \theta x^{(t)}) \cdot x^{(t)}
\end{align*}


\begin{itemize}
\item initialize $\theta=0$
\item randomly pick $t=\{1,\cdots,n\}$
\item $\theta = \theta + \eta(y^{(t)} - \theta x^{(t)}) \cdot x^{(t)}$
\end{itemize}

Where $\eta$ is the learning rate (steps) and the learning rate gets smaller the closer you get $\eta_k = \frac{1}{1+k}$ 


\subsection{Source of mistakes:}
\begin{enumerate}
	\item \textbf{structural mistakes} Maybe the linear function is not sufficient for you to model
your training data. Maybe the mapping between your training vectors and y’s is
actually highly nonlinear. Instead of just considering linear mappings, you should
consider a much broader set of function. This is one class of mistakes. 
	\item \textbf{estimation mistakes} The mapping itself is indeed linear, but we don’t have
enough training data to estimate the parameters correctly.
\end{enumerate}




\subsection{Ridge Regression}
Regularization is trying to push away from perfect fit.

\begin{align*}
J_{n, \lambda } (\theta , \theta _0) &= \frac{1}{n} \sum _{t=1}^{n} \frac{(y^{(t)} - \theta \cdot x^{(t)}-\theta _0)^2}{2} + \frac{\lambda }{2} \left\|  \theta  \right\| ^2\\
\nabla_{\theta}(J_{n, \lambda })&= \lambda \theta - (y^{(t)}-\theta x^{(t)})x^{(t)}
\end{align*}

\begin{itemize}
\item initialize $\theta=0$
\item randomly pick $t=\{1,\cdots,n\}$
\item $\theta = \theta + \eta\lambda \theta - (y^{(t)}-\theta x^{(t)})x^{(t)} = (1-\eta \lambda)\theta + \eta(y^{(t)}-\theta \cdot x^{(t)})$
\end{itemize} 


\subsection{The closed form approach with regularisation}
Or Tikhonov regularisation:\\
$\theta = (X^T X + \lambda I)^{-1} X^T Y$



\subsection{Kernels}

Kernels is a measure of proximity. We transform data into higher dimensions and easily classify
them there(use SVM for example). Idea that we build new features from existing one within higher dimensions.
New coordinates should be linearly independent.\\

Given $x=(x_1, x_2)$ and $x^\prime =(x_1^\prime , x_2^\prime )$
\begin{align*}
\displaystyle  \phi (x) \displaystyle &= \big [x_1, \, x_2,\,  {x_1}^2,\,  \sqrt{2}x_1x_2,\,  {x_2}^2 \big ]\\
\displaystyle \phi (x') \displaystyle &= \big [x_1^\prime ,\,  x_2^\prime ,\,  {x_1^\prime }^2,\,  \sqrt{2}x_1^\prime x_2^\prime ,\,  {x_2^\prime }^2 \big ]\\
\displaystyle  \phi (x) \cdot \phi (x^\prime )&=\displaystyle {x_1}{x_1^\prime } + {x_2}{x_2^\prime } + {x_1x_1^\prime }^2 + 2{x_1}{x_1^\prime }{x_2}{x_2^\prime } + {x_2x_2^\prime }^2\\
&=\displaystyle \left({x_1}{x_1^\prime } + {x_2}{x_2^\prime }\right)+ \left({x_1}{x_1^\prime } + {x_2}{x_2^\prime }\right)^2\\
&=\displaystyle \left({x_1}{x_1^\prime } + {x_2}{x_2^\prime }\right)+ \left({x_1}{x_1^\prime } + {x_2}{x_2^\prime }\right)^2
\end{align*}



\subsection{Kernel Perceptron}

The parameter vector of a preceptron algorithm can also be written as: 
\begin{align*}
\displaystyle  \displaystyle \theta = \sum _{j=1}^{n} \alpha _ j y^{(j)} \phi \left(x^{(j)}\right)
\end{align*}

Where $\alpha_j$ represents the number of classification mistakes the perceptron algo made. Every time a missclassification happens the parameter vector is updated with the product of the label and the feature vector $\theta  = \theta + y^{(i)} x^{(i)}$. The goal of the Kernel Perceptron algo is to find the vector $\alpha$ with the counts of the missclassifications.

\textbf{Kernel Perceptron}$\displaystyle \left(\big \{ (x^{(i)}, y^{(i)}), i=1,...,n, T \big \} \right)$
\begin{enumerate}[\indent {}]
	\item initialize  $\alpha _1, \alpha _2, ..., \alpha _ n;$ to some values
	\begin{enumerate}[\indent {}]
		\item for $t=1,\cdots,T$ do
		\begin{enumerate}[\indent {}]
			\item for $i=1,\cdots,n$ do
			\begin{enumerate}[\indent {}]
				\item if $y^{(i)}\sum _{j=1}^{n} \alpha _ j y^{(j)} K(x^{j},x^{i}) \leq 0$ then
				\item update $\alpha _ i = \alpha _ i +1 y^{(i)}$
			\end{enumerate}
		\end{enumerate}
	\end{enumerate}
\end{enumerate}

\subsection{Polynomial kernel}
$(a \cdot b+r)^d$ -- where $r$ is shift and $d$ is new dimension. $r$ and $d$ parameters which
could be determined from cross-validation.

\subsection{Radial basis Kernel}
\begin{align*}
K(x,x') = e^{-\gamma {||x-x'||}^2}
\end{align*}

RBF founds SVM(?) classificator in infinite dimension space. 
$\gamma$ is tuned parameter searched in cross-validation.
$\gamma$ -- scale squred distance, scale influence.